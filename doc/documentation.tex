% Encoding: utf-8

\documentclass[a4paper, 10pt, oneside, twocolumn, openany]{memoir}

\usepackage{lipsum} % for tests

\usepackage{lmodern}
\usepackage[T1]{fontenc}
\usepackage[utf8]{inputenc}
\usepackage{float} 

\usepackage{textcomp}
\usepackage{amsmath}
\usepackage{amssymb}

\usepackage[pdftex]{graphicx}
\usepackage[colorlinks, pdftex]{hyperref}
\hypersetup{pdftitle={Berreman4x4}}

\usepackage[numbers, super, sort&compress]{natbib}

\settypeblocksize{56pc}{42.5pc}{*}  % text body
\setlrmargins{*}{*}{0.7}            % right / left margin ratio
\setulmargins{*}{*}{2}              % upper / lower margin ratio
\setcolsepandrule{1.5pc}{0pt}       % two column separation and rule
\checkandfixthelayout

\chapterstyle{komalike}
\setlength{\beforechapskip}{0pt}    % vertical space added above chapter name
\setcounter{secnumdepth}{3}

% Numbering without chapter number
\renewcommand*{\thesection}{\arabic{section}}
\renewcommand*{\theequation}{\arabic{equation}}
\renewcommand*{\thefigure}{\arabic{figure}}

% Special command for chapters
\newcommand{\chapterdoc}[1]{\chapter*{#1}%
    \addtocounter{chapter}{1}%
    \setcounter{section}{0}
}

% Chapterprecis settings
\makeatletter
\renewcommand{\prechapterprecis}{%
    \vspace*{\prechapterprecisshift}%
    \begingroup\precisfont}
\renewcommand{\postchapterprecis}{\endgroup\vspace{1em}}
\newcommand{\chapterauthor}[1]{\chapterprecis{#1}
    \@afterindentfalse\@afterheading}
\makeatother
\renewcommand*{\precisfont}{\normalfont\sffamily}


% Provide command name \onlinecite{}
\def\onlinecite{\citenum}

% Caption for figures
\renewcommand*{\figurename}{FIG.}

% Useful definitions
\newcommand{\eps}{\ensuremath{\varepsilon}}

% A few changes in the default commands
\renewcommand{\Re}{\mathop{\mathrm{Re}}}
\renewcommand{\Im}{\mathop{\mathrm{Im}}}





\begin{document}

\begin{titlingpage}

\pretitle{\begin{center}\Huge\sffamily\bfseries}
\posttitle{\par\end{center}\vskip 3.5cm}

\preauthor{\large \lineskip 0.5em \begin{tabular}[t]{l}}
\postauthor{\end{tabular}\par}

\predate{\large \lineskip 0.5em \begin{tabular}[t]{l}}
\postdate{\end{tabular}\par\vskip 1cm}

\usethanksrule
\thanksheadextra{(}{)}
\thanksmarkstyle{\textsuperscript{(#1)}}
\thanksmarkseries{alph}

\setlength{\absparindent}{0pt}
% Use same alignment for abstrat title than for abstrat text
\makeatletter\renewcommand{\absnamepos}{@bstr@ctlist}\makeatother
\renewcommand{\abstractnamefont}{\normalfont\bfseries}
\renewcommand{\abstracttextfont}{\normalfont}

\title{Berreman4x4}

\author{\textsf{Olivier Castany}%
    \thanks{Electronic mail: 
    \href{mailto:olivier.castany@telecom-bretagne.eu}%
         {olivier.castany@telecom-bretagne.eu}}\\
    \emph{Department of Optics, Telecom Bretagne, 29238 Brest, France}
}

\date{\today}

\maketitle

\begin{abstract}
Electromagnetic plane wave propagation in stratified anisotropic media was described by Berreman with the use of 4$\times$4 matrices.
Berreman4x4 is a numerical implementation of the method in Python.
Examples of applications are ellipsometry analysis, design of Bragg mirrors or study of twisted liquid crystal structures.
\end{abstract}

\end{titlingpage}


%%%%%%%%%%%%%%%%%%%%%%%%%%%%%%%%%%%%%%%%%%%%%%%%%%%%%%%%%%%%%%%%%%%%%%%%%%%%%

\input{description}
\input{validation-Fujiwara}
% Encoding: utf-8

\chapterdoc{Example of frustrated total internal reflection}
\chapterauthor{Céline Molinaro, Olivier Castany}


We reproduced the situation of frustrated total internal reflection. A general and theoretical approach is presented first with simple cases. Then an application with the frustrated internal reflection is presented.


\section{Presentation}

As described on figure~\ref{fig:FTIR}, we consider three dielectrics : front (refractive index $n_f$), air (refractive index $n$ and thickness $d$) and back (refractive index $n_b$).

Without loss of generality, we consider plane waves in the $(x,z)$ direction, i.e. $k_y=0$. The front half-space is isotropic and a plane wave can be decomposed into $s$ and $p$ polarizations.
The $s$ polarization is a wave with perpendicular electric field (\emph{senkrecht}) to the plane of incidence, i.e. along $y$.
The $p$ polarization is a wave with parallel electric field to the plane of incidence.
A plane wave $i$ is incident from the front half-space with incidence angle $\phi_i$ and reflected into a plane wave $r$ with the same angle. Angles are oriented by the $y$ direction. 
A total internal reflection occurs if the angle of incidence is larger than the critical angle
$\phi_i > \phi_{ic}$ with $\phi_{ic}=\arcsin\displaystyle\frac{n}{n_f}$

An evanescent wave is obtained in the adjacent medium. The third medium allows the frustrated total internal reflection. A transmitted plane wave $t$ occurs with an angle $\phi_t$.

\begin{figure}[!h]
\includegraphics[width=\columnwidth]{fig/FTIR}
\caption{\label{fig:FTIR}Frustrated total internal reflection with input and output plane waves. }
\end{figure}
 
\section{Poynting vector in general cases}
This section contains a description of three general cases whose can be applied to our system. In each description, we are interested in the Poynting vector, which is defined by:
$$\langle \vec{\Pi} \rangle _t = \begin{pmatrix}
\langle \Pi _x \rangle _t\\
\langle \Pi _z \rangle _t
\end{pmatrix}
=\frac{\Re(\vec{E}\times \vec{H}^*)}{2} $$
In the case of a polarization $s$, we get $\vec{H}$ from $\vec{E}$ from using a Maxwell equation
$\vec{\nabla}\times\vec{E} = -\mu_0 \frac{\partial\vec{H}}{\partial t}$.\\
In the case of a polarization $p$, we get $\vec{E}$ from $\vec{H}$ from using a Maxwell equation $\vec{\nabla}\times\vec{H}=\epsilon_0\frac{\partial\vec{E}}{\partial t}$.\\

\subsection{Plane wave}
In this case, we consider one wave in its isotropic medium with a refractive index $n$ (like the wave in the back space for the frustrated total internal reflection). Calculations will be done for both polarizations $s$ and $p$ in order to find the Poynting vector.\\

\subsubsection{Polarization $s$}
The electric field is $\vec{E}=E_0e^{-i\omega t+i(k_zz+k_xx)}\vec{y}$ where $k_x\in \mathbb{R}$ and $k_z\in \mathbb{C}$.
The magnetic field is\\
\begin{equation*}
\vec{H}^*=\frac{E_0^*e^{i\omega t-i(k_z^*z+k_xx)}}{\omega \mu _0}
\begin{pmatrix}
-k_z^*\\
k_x
\end{pmatrix}
\end{equation*}
 with $k_z=k_z'+ik_z''$.\\
The Poynting vector is:
\begin{equation}
\langle \vec{\Pi} \rangle _t=\displaystyle\frac{\displaystyle\mid E_0\mid ^2e^{-2k_z''z}}{\displaystyle2\omega \mu _0}
\begin{pmatrix}
k_x\\
k_z'
\end{pmatrix}
\end{equation}

\subsubsection{Polarization $p$}
The magnetic field is:\\
$$\vec{H}=H_0e^{-i\omega t+i(k_zz+k_xx)}\vec{y}$$\\
The electric field is:
\begin{equation*}
\vec{E}=\frac{H_0e^{-i\omega t+i(k_zz+k_xx)}}{\omega\epsilon_0}
\begin{pmatrix}
k_z\\
-k_x
\end{pmatrix}
\end{equation*}
The Poynting vector is:
\begin{equation}
\langle \vec{\Pi} \rangle _t=\displaystyle\frac{\displaystyle\mid H_0\mid ^2e^{-2k_z''z}}{\displaystyle2\omega \epsilon _0}
\begin{pmatrix}
k_x\\
k_z'
\end{pmatrix}
\end{equation}

\subsubsection{Physical interpretation}
In both cases of polarization, if $k_z\in \mathbb{R}$, the Poynting vector is oriented along the wave vector and its magnitude is constant. If $k_z\in i\mathbb{R}$, the Poynting vector is parallel to the real part of the wave vector, and its magnitude decreases exponentially along $z$. Poynting vectors and wave vectors are represented on figure~\ref{fig:Plane_wave_real_complex}.\\
\begin{figure}[h!]
\includegraphics[width=\columnwidth]{fig/Plane_wave_real_complex}
\caption{\label{fig:Plane_wave_real_complex}Plane  wave representation when $k_z$ is real (a) and when $k_z$ is an imaginary number (b)}
\end{figure}

\subsection{Plane wave reflected by an arbitrary medium}
In this case there are two waves, the incident and the reflected. The incident medium which is isotropic is the most interesting one. The only interesting thing for the right medium is its reflection coefficient.
\subsubsection{Polarization $s$}
$\vec{E}^+$ is the incident wave and $\vec{E}^-$ is the refracted wave. The electric field in the incident medium is
\begin{equation*}
\vec{E}=\vec{E}^++\vec{E}^-
\end{equation*}
The relations between the amplitudes of the waves
\begin{equation*}
E^-_0=r_sE^+_0
\end{equation*}
where $r_s=r_s'+ir_s''$ is the amplitude coefficient reflection.\\
The electric field is given by $$\vec{E}=E^+_0(e^{-i\omega t+i(k_xx+k_zz)}+r_se^{-i\omega t+i(k_xx-k_zz)})\vec{y}$$
where $k_x\in \mathbb{R}$ and $k_z\in \mathbb{C}$.\\
The magnetic field is\\
\begin{align*}
\vec{H}^*=\frac{E^{+*}_0}{\omega \mu _0}(&k_z^*(-e^{i\omega t-i(k_xx+k_z^*z)}+r_s^*e^{i\omega t+i(k_zz-k_xx)})\vec{x}\\
&+k_x(e^{i\omega t-i(k_xx+k_z^*z)}+r_s^*e^{i\omega t+i(k_zz-k_xx)})\vec{z}
\end{align*}\\
The Poynting vector is:\\
\begin{align*}
\langle \vec{\Pi} \rangle _t= \displaystyle\frac{\displaystyle \mid E^+_0\mid ^2}{\displaystyle 2\omega \mu _0}\Big\{&
e^{-2k_z''z}
\begin{pmatrix}
k_x\\
k_z'
\end{pmatrix}
+\mid r_s\mid ^2e^{2k_z''z}
\begin{pmatrix}
k_x\\
-k_z'
\end{pmatrix}\\
&+2
\begin{pmatrix}
k_xr_s'\cos(2k_z'z)\\
k_z''r_s''\sin(2k_z'z)
\end{pmatrix}
\Big\}
\end{align*}

\subsubsection{Polarization $p$}
$\vec{H}^+$ is the incident wave and $\vec{H}^-$ is the refracted wave. The magnetic field is
\begin{equation*}
\vec{H}=\vec{H}^++\vec{H}^-
\end{equation*}
The relation between these two components is
\begin{equation*}
H^-_0=r_pH^+_0
\end{equation*}
where $r_p=r_p'+ir_p''$ is the amplitude coefficient reflection.\\
The electric field is given by $$\vec{H}=H^+_0(e^{-i\omega t+i(k_xx+k_zz)}+r_pe^{-i\omega t+i(k_xx-k_zz)})\vec{y}$$
where $k_x\in \mathbb{R}$ and $k_z\in \mathbb{C}$.\\
The electric field is\\
\begin{align*}
\vec{E}=\frac{H^{+}_0}{\omega \epsilon _0}(&k_z^*(e^{-i\omega t+i(k_xx+k_zz)}-r_pe^{-i\omega t+i(k_xx-k_zz)})\vec{x}\\
&-k_x(e^{-i\omega t+i(k_xx+k_zz)}+r_pe^{-i\omega t+i(k_xx-k_zz)})\vec{z}
\end{align*}
The Poynting vector is:\\
\begin{align*}
\langle \vec{\Pi} \rangle _t= \displaystyle\frac{\displaystyle \mid H_0\mid ^2}{\displaystyle 2\omega \epsilon _0}\Big\{&
e^{-2k_z''z}
\begin{pmatrix}
k_x\\
k_z'
\end{pmatrix}
+\mid r_p\mid ^2e^{2k_z''z}
\begin{pmatrix}
k_x\\
-k_z'
\end{pmatrix}\\
&+2
\begin{pmatrix}
k_xr_p'\cos(2k_z'z)\\
k_z''r_p''\sin(2k_z'z)
\end{pmatrix}
\Big\}
\end{align*}
We found the same result in the previous subsection. The reflection coefficient is the only different thing.
\subsubsection{Physical interpretation}
In both cases of polarization, if $k_z\in \mathbb{R}$, the Poynting vector is constant along $z$. There are interferences along $x$ given by $k_x\cos(\theta _{r'}-2k_z'z)$.
If $k_z\in i\mathbb{R}$, there are interferences along both $x$ and $z$ directions given by $k_xr_{s,p}'\cos(2k_z'z)$ and $k_z''r_{s,p}''\sin(2k_z'z)$ respectively. Along $z$ there is only an interference term.
\begin{figure}[!h]
\includegraphics[width=\columnwidth]{fig/Wave_evanescent_poynting}
\caption{\label{fig:Wave_evanescent_poynting}Poynting vector for evanescent wave 
 and for their superimpose}
\end{figure}
This case is represented on figure ~\ref{fig:Wave_evanescent_poynting}. Each wave has a Poynting vector along $x$. When they are superimpose, interferences occurs on $z$, so the Poynting vector has a component along $z$ too. The Poynting vector addition is not linear.

\subsection{Plane wave reflected by a diopter}
An incident wave and its reflective wave are in the first medium with a refractive index $n_1$. The waves can be plane or evanescent. In the second medium with a refractive index $n_2$ there is a refracted or transmitted plane wave.  
\subsubsection{Polarization $s$}
The amplitude coefficient reflection for the electric field is known:
$$r_s = \frac{k_{z1}-k_{z2}}{k_{z1}+k_{z2}} \ \ \text{and} \ \ t_s=1+r_s$$
The calculations are done in the most general case so $k_{z1}\in \mathbb{C}$ and $k_{z2}\in \mathbb{C}$.\\
$k_{z1}=k_{z1}'+ik_{z1}''$ and $k_{z2}=k_{z2}'+ik_{z2}''$
\begin{equation*}
r_s=\frac{\mid k_{z1}\mid ^2-\mid k_{z2}\mid ^2+i2\Im(k_{z1}k_{z2}^*)}{\mid k_{z1}+k_{z2}\mid^2}
\end{equation*}
The interesting part of $r_s$ is its imaginary part:\\
\begin{equation*}
r_s'' = \frac{2(k_{z2}'k_{z1}''-k_{z1}'k_{z2}'')}{\mid k_{z1}+k_{z2}\mid^2}
\end{equation*}
\subsubsection{Polarization $p$}
The amplitude coefficient reflection for the magnetic field is known:
$$
r_p=\frac{\epsilon_2k_{z1}-\epsilon_1k_{z2}}{\epsilon_2k_{z1}+\epsilon_1k_{z2}}
 \ \ \text{and} \ \ t_p = \frac{\cos(\phi_1)(1-r_p)}{\cos(\phi_2)}$$
with $\epsilon_i=n_i^2$ with $i\in\{p,s\}$.
The interesting part of $r_p$ is its imaginary part:\\
\begin{equation*}
r_p'' = \frac{2\epsilon_1\epsilon_2(k_{z2}'k_{z1}''-k_{z1}'k_{z2}'')}{\mid k_{z1}+k_{z2}\mid^2}
\end{equation*}
This part is used to know the sign of the third term in the Poynting vector expression.
\subsubsection{Physical interpretation}
The waves can be plane $k_{z}\in \mathbb{R}$ or evanescent $k_{z}\in i\mathbb{R}$.


\section{Application to the frustrated internal total reflection}

\subsection{Electric fields and wave vectors}
To study the frustrated total internal reflection, calculations are done in the case where the angle of incidence is larger than the critical angle:
$\phi_i > \phi_{ic}$
Five waves are studied. Each wave has the same wave vector $k_x$.
\begin{center}
$k_x=n_fk_0\sin(\phi_i)$ and $k_x=n_bk_0\sin(\phi_t)$\\
\end{center}

The first two are in the front medium where the wave vector along $z$, $k_{fz}\in \mathbb{R}$.
\begin{itemize}
\item  Incident wave  $\vec{E}_i(x,z)=\vec{E}_i(z)e^{-i\omega t+i(k_{fz}z+k_xx)}$
\item Reflected wave  $\vec{E}_r(x,z)=\vec{E}_r(z)e^{-i\omega t+i(-k_{fz}z+k_xx)}$\\
where $k_{fz}=n_fk_0\cos(\phi _i)$.
\end{itemize}
Two others are in the air where $k_{nz}\in i\mathbb{R}$. One is the transmitted wave from the front medium and the other is the reflected wave.
\begin{itemize}
\item Incident wave 
$\vec{E}_{ni}(x,z)=\vec{E}_ni(z)e^{-i\omega t+i(k_{nz}z+k_xx)}$
\item Reflected wave 
$\vec{E}_{nr}(x,z)=\vec{E}_nr(z)e^{-i\omega t+i(-k_{nz}z+k_xx)}$\\
where $k_{nz}=ik_0\sqrt{n_f^2\sin^2(\phi _i)-n^2}$.
\end{itemize}
The last wave is in the back medium where $k_{b}\in \mathbb{R}$.
\begin{itemize}
\item Transmitted wave
$\vec{E}_t(x,z)=\vec{E}_t(z)e^{-i\omega t+i(k_{bz}z+k_xx)}$\\
where $k_{bz}=n_bk_0\cos(\phi _t)$.
\end{itemize}
In this medium the plane wave has an angle \\
$$
\phi_t=\arcsin (\frac{n_{fz}\sin(\phi_i)}{n_b})
$$

\subsection{Continuity equations} 
The electric waves follow the $y$ axe.\\
Continuity equations $z=0$
\begin{equation}
\left\lbrace
\begin{array}{ccc}\label{eq:continuity0}
E_{ni}&=r_{fn}E_{nr}+t_{nf}E_i\\
E_r&=r_{nf}E_{i}+t_{fn}E_{nr}
\end{array}\right.
\end{equation}
Continuity equations $z=z_b$
\begin{equation}
\left\lbrace
\begin{array}{ccc}\label{eq:continuityd}
E_t=t_{bn}E_{ni}\\
E_{nr}=r_{bn}E_{ni}
\end{array}\right.
\end{equation}
$E_{ni}$ and $E_{nr}$ can be written at $z=0$ or $z=z_b$. The relations between them are
\begin{equation}
\left\lbrace
\begin{array}{ccc}\label{eq:0tod}
E_{ni}(0)=E_{ni}(d)e^{-ik_{nz}d}\\
E_{nr}(0)=E_{nr}(d)e^{ik_{nz}d}
\end{array}\right.
\end{equation}

With the equations~\eqref{eq:continuity0},~\eqref{eq:continuityd} and~\eqref{eq:0tod}, we got\\
\begin{equation}
E_{ni}(0)=\frac{t_{nf}}{1+r_{nf}r_{bn}e^{i2k_{nz}d}}E_i(0)
\end{equation}

\begin{equation}
E_{nr}(0)=\frac{t_{nf}t_{bn}e^{i2k_{nz}d}}{1+r_{nf}r_{bn}e^{i2k_{nz}d}}E_i(0)
\end{equation}

\begin{equation}\label{eq:reflectivewave}
E_{r}(0)=\frac{r_{fn}+r_{bn}e^{i2k_{nz}d}}{1+r_{nf}r_{bn}e^{i2k_{nz}d}}E_i(0)
\end{equation}

\begin{equation}\label{eq:transmittedwave}
E_{t}(z_b)=\frac{t_{nf}t_{bn}e^{ik_{nz}d}}{1+r_{nf}r_{bn}e^{i2k_{nz}d}}E_i(0)
\end{equation}
According to equation~\eqref{eq:reflectivewave}, the amplitude coefficient reflection for the interface is\\
$$
r=\frac{r_{fn}+r_{bn}e^{i2k_{nz}d}}{1+r_{nf}r_{bn}e^{i2k_{nz}d}}
$$
The reflection coefficient is given by
\begin{equation*}
R=\mid r\mid^2
\end{equation*}
According to equation~\eqref{eq:transmittedwave}, the amplitude coefficient transmission for the interface is\\
$$
t=\frac{t_{nf}t_{bn}e^{ik_{nz}d}}{1+r_{nf}r_{bn}e^{i2k_{nz}d}}
$$
The transmission coefficient is given by
\begin{equation*}
T=\frac{n_b\cos\phi _t}{n_f\cos\phi _i}\mid t\mid^2
\end{equation*}
The conservation of energy $R+T=1$ is respected in the cases of polarization $p$ and polarization $s$.


\subsection{Front medium}
In this medium, there are two waves, the incident and the reflected. The total electric field is:
$\vec{E}_f = \vec{E}_i+\vec{E}_r$\\
On the $y$ axe:
\begin{eqnarray*}
E_f = E_i(0)\ 
\big\{&  e^{-i\omega t+i(k_{fz}z+k_xx)} + \cdots  \\
 &+ r_{tot} \ e^{-i\omega t+i(-k_{fz}z+k_xx)} \big\}
\end{eqnarray*}
with 
\begin{eqnarray*}
r_{tot} & = & \frac{r_{fn}+r_{bn}e^{i2k_{nz}d}}{1+r_{nf}r_{bn}e^{i2k_{nz}d}}\\
& = & r_{tot}'+ir_{tot}''
\end{eqnarray*}
%The conugate magnetic field is:\\
%\begin{eqnarray*}
%\vec{H}^*=\frac{E_i(0)^*}{\omega \mu_0}\{&k_{fz}(-e^{i\omega t-i(k_{fz}z+k_xx)}\\
%+r_{tot}^*e^{i\omega t+i(k_{fz}z-k_xx)})\vec{x}\\
%+&k_x(e^{i\omega t-i(k_{fz}z+k_xx)}\\
%+r_{tot}^*e^{i\omega t+i(k_{fz}z-k_xx)})\vec{z}\}
%\end{eqnarray*}
The Poynting vector:\\
\begin{align*}
\langle \vec{\Pi} \rangle _t =\displaystyle\frac{\displaystyle\mid E_i(0)\mid ^2}{\displaystyle2\omega \mu_0}\Big\{&
\begin{pmatrix}
k_x\\
k_{fz}
\end{pmatrix}
+\mid r_{tot}\mid^2
\begin{pmatrix}
k_x\\
-k_{fz}
\end{pmatrix}\\
&+2
\begin{pmatrix}
r_{tot}'\cos(2k_{fz}z\\
0
\end{pmatrix} 
\Big\}
\end{align*}

\subsection{Air medium}
In this medium, there are two waves. The total electric field is:
$\vec{E}_n=\vec{E}_{ni}+\vec{E}_{nr}$\\
On the $y$ axe:\\
\begin{eqnarray*}
E_n = E_{ni}(0)\ 
\big\{&  e^{-i\omega t+i(k_{nz}z+k_xx)} + \cdots  \\
 &+ r_n \ e^{-i\omega t+i(-k_{nz}z+k_xx)} \big\}
\end{eqnarray*}
with:
\begin{eqnarray*}
r_n & = & r_{bn}e^{i2k_{nz}d}\\
& = & r_n'+ir_n''
\end{eqnarray*}

%The conjugate magnetic fiels has been calculated in the most general case, with $k_{nz}\in \mathbb{C}$:\\
%\begin{eqnarray*}
%\vec{H}^*=\frac{E_{ni}(0)^*}{\omega \mu_0}\{
%&k_{nz}^*\big(-e^{i\omega t-i(k_{nz}^*z+k_xx)}\\
%+r_n^*e^{i\omega t+i(k_{nz}^*z-k_xx)}\big)\vec{x}\\
%+&k_x\big(e^{i\omega t-i(k_{nz}^*z+k_xx)}\\
%+e^{i\omega t+i(k_{nz}^*z-k_xx)}\big)\vec{z}\}
%\end{eqnarray*}
%We write:\\
%$$
%k_{nz}=k_{nz}'+ik_{nz}''
%$$
%The Poynting vector is:\\
%$
%\langle \vec{\Pi} \rangle _t =\displaystyle\frac{\displaystyle\mid E_{ni}(0)\mid ^2}{\displaystyle2\omega \mu_0}\Big\{
%e^{-2k_{nz}''z}
%\begin{pmatrix}
%k_x\\
%k_{nz}'
%\end{pmatrix}\\
%+\mid r_n\mid ^2e^{2k_{nz}''z}
%\begin{pmatrix}
%k_x\\
%-k_{nz}'
%\end{pmatrix}\\
%+2\mid r_n\mid
%\begin{pmatrix}
%k_x\cos(\theta _n-2k_{nz}'z)\\
%k_{nz}''\sin(\theta _n -2k_{nz}'z)
%\end{pmatrix}
%\Big\}
%$\\
In the case of the frustrated total internal reflection, the wave in the air medium is an evanescent wave, $k_{nz} \in i\mathbb{R}$. The Poynting vector is:\\
$
\langle \vec{\Pi} \rangle _t =\displaystyle\frac{\displaystyle\mid E_{ni}(0)\mid ^2}{\displaystyle2\omega \mu_0}\Big\{
e^{2k_{nz}z}
\begin{pmatrix}
k_x\\
0
\end{pmatrix}
+\mid r_n\mid ^2e^{-2k_{nz}z}
\begin{pmatrix}
k_x\\
0
\end{pmatrix}\\
+2
\begin{pmatrix}
k_xr_n'\\
ik_{nz}^*r_n''
\end{pmatrix}
\Big\}
$\\
The third term is an interference terms. Along $z$ we get:\\
$$
ik_{nz}^*\sin(\theta _n) =\frac{ k_0^3n_b\cos(\phi _t)(n_f^2\sin ^2(\phi _i)-n^2)e^{ik_{nz}d}}{\mid k_{nz}+k_{fz}\mid ^2}
$$
This expression is positive so the interference term is oriented to the positive $z$.
\subsection{Back medium}
In this medium, there is one wave, the transmitted wave.With equation~\eqref{eq:transmittedwave}, on the $y$ axe:\\
\begin{equation*}
E_t = E_i(0)te^{-i\omega t+i(k_bz+k_xx)}
\end{equation*}
with:
$$
t = \frac{t_{nf}t_{bn}e^{ik_{nz}d}}{1+r_{nf}r_{bn}e^{i2k_{nz}d}}
$$
%The conjugate magnetic field is:\\
%\begin{eqnarray*}
%\vec{H}^*= \frac{E_i(0)^*}{\omega \mu_0}\Big\{-k_t^*be^{i\omega t-i(k_bz+k_xx)}\vec{x}\\
%+k_xt^*e^{i\omega t-i(k_bz+k_xx)}\vec{z}\Big\}
%\end{eqnarray*}
The Poynting vector is:\\
$$
\langle \vec{\Pi} \rangle _t = \frac{\mid E_i(0)\mid ^2\mid t\mid ^2}{2\omega \mu_0}
\begin{pmatrix}
k_x\\
k_b
\end{pmatrix}
$$

% Encoding: utf-8

\chapterdoc{Description of the examples}
\chapterauthor{Céline Molinaro, Olivier Castany, }

\section{Reflection at an interface}
This is a simple case as shown on figure~\ref{fig:Reflection-interface}. An incident wave is reflected and refracted by an interface. There are two mediums with different refractive index $n_1$ and $n_2$. This interface is characterized by the transmission and reflection power coefficient.
\begin{figure}[H]
\includegraphics[width=\columnwidth]{fig/Reflection-interface}
\caption{\label{fig:Reflection-interface}Input and output plane waves at the interface.}
\end{figure}
In \verb/interface-Jones.py/, the Jones matrix for the interface are calculated. In \verb/ interface-reflection.py/, transmission and reflection power coefficients are calculated and drawn for both polarizations $s$ and $p$ vs. wave vector. Theoretical transmission and reflection power coefficients are drawn on the same graphic. See source code for further details.

\section{Glass layer}
This is a simple case as shown on figure~\ref{fig:Glass-layer}. An incident wave is reflected and transmitted by a layer. Each medium is characterized by its refractive index.
\begin{figure}[H]
\includegraphics[width=\columnwidth]{fig/Glass-layer}
\caption{\label{fig:Glass-layer}Incident, reflected and transmitted waves in a glass layer case.}
\end{figure}
In \verb/Interferences.py/, transmission and reflection power coefficients for $p$ and $s$ polarization are drawn vs. the glass layer thickness. Interferences appear depending on the layer thickness value.

\section{Bragg Mirror}
A Bragg mirror with two layers is represented on figure~\ref{fig:Bragg}. Each layer is composed of two thin dielectric layers: SiO$_2$ and TiO$_2$. An incident wave, its reflected waves and its transmitted wave are represented. Bragg mirrors are used to produce ultra-high reflectivity.
\begin{figure}[H]
\includegraphics[width=\columnwidth]{fig/Bragg}
\caption{\label{fig:Bragg}Bragg Mirror with two layers}
\end{figure}
In \verb/Bragg.py/, the Bragg mirror has $8,5$ periods. The reflection and transmission power coefficients for a $s$ polarization are drawn vs. the wavelength. In \verb/validation-Bragg.py/, theoretical reflection coefficients for both polarization are superimpose to the calculated global reflection coefficients depending on the wave length. Reflection coefficients are valued for two different incidence angles (0$^\circ$ and 45$^\circ$). The nombre of layers and the spectra can be modified by the user. See source code for further details.

\section{Twisted nematic liquid crystal}
The twisted crystal is aligned along $x$ and twists with  $90^\circ$ between $z=0$ and $z=d$ as shown on figure~\ref{fig:nematic}.
\begin{figure}[H]
\begin{center}
\includegraphics{fig/nematic}
\caption{\label{fig:nematic} Twisted nematic crystal}
\end{center}
\end{figure}
In \verb/twisted-nematic.py/, the power transmission for two different numbers of division in the twisted material and a theoretical are drawn vs. the wave number. See source code for further details. 
 
\section{Cholesteric liquid crystal}
The characteristic of a cholesteric liquid crystal as shown on figure~\ref{fig:cholesteric} is its helical structure. Its director vector is rotated. The pitch represents the period of the rotation variation.
This example deals with a $n$ layers cholesteric liquid crystal.
\begin{figure}[H]
\begin{center}
\includegraphics{fig/cholesteric}
\end{center}
\caption{\label{fig:cholesteric}A cholesteric liquid crystal where p is the pitch}
\end{figure}
In \verb/validation-cholesteric.py/, the calculated and theoretical reflection spectra of the $n$ cholesteric layers are drawn. The theoretical expression used to draw a theoretical reflection power coefficient comes from \cite{Wu} and \cite{Chandrasekhar}. See source code for further details. In \verb/cholesteric.py/, transmission (for different polarizations: right-circular, $p$, depolarized...) and reflection spectra are drawn. Eigenvalues and eigenvectors of the transmission matrix for different cases are calculated. See source code for further details. 


%%%%%%%%%%%%%%%%%%%%%%%%%%%%%%%%%%%%%%%%%%%%%%%%%%%%%%%%%%%%%%%%%%%%%%%%%%%%%

\bibliographystyle{unsrtnat}
\bibliography{documentation}

\end{document}

